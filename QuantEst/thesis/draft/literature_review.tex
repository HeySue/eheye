\documentclass[12pt]{article}
\usepackage{xcolor}
\usepackage[margin=1in]{geometry}
\usepackage{amssymb}
\usepackage{amsmath}
\usepackage{bm}
\usepackage{xcolor}

\usepackage[english]{babel}
\usepackage[utf8]{inputenc}
\usepackage{algorithm}
% \usepackage{algpseudocode}
% \usepackage{algorithmic}
\usepackage[noend]{algpseudocode}
\usepackage{enumitem}


\def\R{\rm I\!R}
\def\x{\bm{x}}
\DeclareMathOperator*{\argmax}{arg\,max}
\DeclareMathOperator*{\argmin}{arg\,min}

\renewcommand{\algorithmicrequire}{\textbf{Input:}}
\renewcommand{\algorithmicensure}{\textbf{Output:}}
\usepackage[numbers]{natbib}

\title{Literature review outline}
\date{\vspace{-5ex}}


\begin{document}
\maketitle

\section{SGD 
            \color{blue}{(should it be in Background instead?)}}
    

\section{Quantile Estimation From Streaming Data}
\begin{enumerate}
    \item Why useful: \\
        \cite{rayArtApproximatingDistributions1800}(Industrial use) "
        Many businesses care about accurately computing quantiles over their key metrics, which can pose several interesting challenges at scale. 

        e.g. Price for advertisement on bidding level: quantile estimation helps price setting
        "\\\\
        Other industrial usage mentioned in \cite{hongEstimatingQuantileSensitivities2009}, and its citations for industrial use \\
        "
        Quantiles have been adopted by many industries as major
        measures of random performance. In the financial industry,
        quantiles, also known as value-at-risks (VaRs), are widely
        accepted measures of capital adequacy. For example, the
        Bank for International Settlement uses the 10-day VaR at
        the 99\% level to measure the adequacy of bank capital
        (Duffie and Pan 1997). In the service industry, quantiles
        are often used as measures of service quality. For example, the service quality of an out-of-hospital system is frequently measured by the 90th percentile of the times taken
        to respond to emergency req`uests and to transport patients
        to a hospital (Austin and Schull 2003). Quantiles have
        also been used as billing measures in some circumstances.
        For example, some Internet service providers (ISPs) charge
        their users based on the 95th percentile of the traffic load
        in a billing cycle (Goldenberg et al. 2004).
        "
    % \item Pinball loss on quantile regression: \\
    %     \cite{steinwartEstimatingConditionalQuantiles2011}\\
    %     \cite{koenkerRegressionQuantiles1978}
    % \item Simultaneously predicting several quantiles: \\
    %     \cite{sangnierJointQuantileRegression} (non-streaming)(multi-dimensional)(quantile regression)\\
    \item Quantile Streaming: \\
        \cite{greenwaldQuantilesEquidepthHistograms2016}\\
        \cite{maFrugalStreamingEstimating2014}\\
    \pagebreak    
    \item Parallel estimation \textbf{on other things} from Streaming Data: \\
        % \cite{jainP2AlgorithmDynamic1985}\\
        Related work on parallel quantile estimation on streaming data.



        Multiple quantile estimation from streaming data requires the estimation of several different quantile values being calculated simultaneously from streaming data. It has been an issue targeted by different algorithms.\\\\

        \cite{ben-haimStreamingParallelDecision} 
        \textbf{A Streaming Parallel Decision Tree Algorithm}

        The Streaming Parallel Decision Tree (SPDT) algorithm \cite{ben-haimStreamingParallelDecision} introduces an on-line histogram building method % from streaming data at parallel processors.
        in which histogram boundaries are estimated quantile values.
        In this method, multiple histograms are built from streaming data in parallel, which are then merged into a summary histogram of the entire dataset. The summary histogram is a set of sorted real numbers that represents the interval boundaries such that all the intervals have approximately the same size. Specifically, for a summary histogram with $N$ intervals, the set of real numbers is approximately the set of $\tau$-quantiles ($\tau = \frac{1}{N}, \frac{2}{N}, ..., \frac{N-1}{N}$) for the input data stream.

        Advantages: works for distributed system where big data stream is processed by different processors. 
        % This summary histogram is notable for its evenly distributed intervals sizes, as each interval has the same number of data points. To interpret the histogram into quantiles, 
        \\\\
        \cite{pebayFormulasRobustOnepass2008}\\

\end{enumerate}


\section{Anomaly Detection and Outlier}

\begin{enumerate}
    \item Anomaly detection: \\
        \cite{emmottMetaAnalysisAnomalyDetection2015}
        (industrial use)
        ()
        % \cite{huangOnlineAnomalousTime2013}
\end{enumerate}
\newpage
\textbf{My work} Quantile estimation on streaming data, which would be applied on 

\newpage
% \citeauthor{blassWhenAreTwo2008}
\bibliography{Thesis}
\bibliographystyle{IEEEtranN}

\end{document}
\end(documentclass)