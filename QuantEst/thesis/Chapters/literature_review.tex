% \documentclass[12pt]{article}
% \usepackage{xcolor}
% \usepackage[margin=1in]{geometry}
\usepackage{amssymb}
\usepackage{amsmath}
\usepackage{bm}
\usepackage{xcolor}

\usepackage[english]{babel}
\usepackage[utf8]{inputenc}
\usepackage{algorithm}
% \usepackage{algpseudocode}
% \usepackage{algorithmic}
\usepackage[noend]{algpseudocode}
\usepackage{enumitem}


\def\R{\rm I\!R}
\def\x{\bm{x}}
\DeclareMathOperator*{\argmax}{arg\,max}
\DeclareMathOperator*{\argmin}{arg\,min}

\renewcommand{\algorithmicrequire}{\textbf{Input:}}
\renewcommand{\algorithmicensure}{\textbf{Output:}}
% \usepackage[numbers]{natbib}

% \title{Literature review}
% \date{\vspace{-5ex}}


% \begin{document}
% \maketitle
\chapter{Literature review}
\label{ch: literature_review}


The field of quantile estimation on data stream is big, and this thesis is incapable of covering all the topics.
This chapter details some of the work that has been done in this field, and is organised as follows:

Section~\ref{streamingdata} introduces the data stream phenomenon and some quantile estimation methods on it, and are classifies by algorithm types. We specially highlight the space complexity of those algorithms, to show the improvements of quantile estimation algorithms in an intuitive way.

Section~\ref{singlequantile} focuses algorithms with extreme restriction: estimate a quantile in constant space complexity. Two methods are described in detail, followed by a brief comparison.

Since algorithms in section~\ref{singlequantile} cannot deal with multiple quantiles, section~\ref{multiquantile} introduces an improved constant space usage algorithm for quantile summary queries, which also becomes the inspiration of our work.
%----------------------------------------------------------------------------------------
%   Quantile estimation methods on streaming data
%----------------------------------------------------------------------------------------
\iffalse
\section{Quantile estimation methods on streaming data}
\label{streamingdata}
Data stream is the phenomenon that input data comes at a high rate instead of being accessible all at the same time\cite{muthukrishnanDataStreamsAlgorithms2005}.
Generally speaking, the continuous data can come without a start or end, or persistent time intervals.
Streaming data has become popular in industrial application, especially big data use cases, for it
``includes a wide variety of data such as log files generated by customers using your mobile or web applications, ecommerce purchases, in-game player activity, information from social networks, financial trading floors, or geospatial services, and telemetry from connected devices or instrumentation in data centers.''\cite{WhatStreamingData}.

Compared with instantly accessible data, three challenges stand out in data stream: transmission, computation and storage\cite{muthukrishnanDataStreamsAlgorithms2005}.
Transmission problem is difficult when there are frequent delays or data in complicated format,
while challenges in computation comes with high complexity programs. Also, the balance between long-term demanding and capable memory storage can be another tricky problem.
In this research, we focus only on the latter two problems. The aim range is limited to programs with low computation complexity and restricted memory storage, under the assumption that the low-dimensional input data is transmitted at a not very high rate.
\\\\
A common assumption people make about data stream is that the data points are considered as independent individual samples from a certain statistic distribution. 
Although access to the distribution is impossible, studies on the random samples can reveal some useful information. 
Quantile is an important \textbf{feature (?)} for a distribution, which represents the cutting points which divides a data distribution by its ranking.
At the first sight, the computation of quantile is relatively as long as the dataset is sorted.
The data stream, however, is updated at the arrival of each single input, indicating quantile calculation then demands one sorting for every updates.
For what's worse, the unknown size of data stream can cause a severe memory shortage problem for any system. 
The conclusion now becomes clear that quantile calculation fail to be a feasible solution for data stream.

Quantile estimation replaces calculation for data streams due to the computation and storage problems.
The estimation works with limited computational power and memory storage.
Generally, quantile estimation algorithms trade off accuracy for computation and storage.
\citeauthor{munroSelectionSortingLimited1980} \cite{munroSelectionSortingLimited1980} has shown that, for any algorithm that computes the exact $\phi$-quantile from an $N$-element data stream with $p$ passes, requires at least $\Omega(N^{\frac{1}{p}})$ space.
Therefore quantile estimation algorithms all necessarily aim for sub-linear space.
In this chapter, we introduce other people's work on how improvements are made for higher estimation accuracy with lower computational and memory complexity, and that how the trade off is balanced in different situations. Section~\ref{deterministic}, ~\ref{randomised} and ~\ref{other} divide the algorithms by their types: deterministic, randomised and others. The classification is based on works from \citeauthor{buragohainQuantilesStreams2009}\cite{buragohainQuantilesStreams2009} and \citeauthor{wangQuantilesDataStreams2013}\cite{wangQuantilesDataStreams2013}.
% Model: time series: each input $a_i$ equals $A[i]$ and appears in increasing order.
% One-pass application: the reading of an input is done in exactly once and all inputs are read in order. Usually, the process for each input takes only one or few passes. 

It is worth noting that since quantiles are related with ranking, a quantile estimation algorithm takes one-dimensional numerical data input by default. The simple data format also partially explains why transmission is not usually considered a problem in most quantile estimation algorithms.
% -----------------------------------------------------------------------------
% to be finished
\subsection{Deterministic algorithms}
\label{deterministic}
% Manku's work
The first deterministic algorithm on quantile estimation was proposed by \citeauthor{mankuApproximateMediansOther1998}\cite{mankuApproximateMediansOther1998} (referred to as the MRL algorithm below), based on the work by \citeauthor{munroSelectionSortingLimited1980}\cite{munroSelectionSortingLimited1980}. Along with the algorithm, the notation \textit{$\epsilon$-approximate $\phi$-quantile} is first introduced for accuracy measurement by \citeauthor{mankuApproximateMediansOther1998}, which describes the property of guaranteed accuracy on $\phi$-quantile within a pre-specified precision $\epsilon$. The algorithm has a space complexity $O(\frac{1}{\epsilon}\log^2 (\epsilon N))$ to compute \textit{$\epsilon$-approximate quantile summary}. Building on this, two improvement algorithms have been proposed by \citeauthor{greenwaldQuantilesEquidepthHistograms2016a}\cite{greenwaldQuantilesEquidepthHistograms2016a} and \citeauthor{shrivastavaMediansNewAggregation2004b}\cite{shrivastavaMediansNewAggregation2004b}.

\citeauthor{greenwaldQuantilesEquidepthHistograms2016a}\cite{greenwaldQuantilesEquidepthHistograms2016a} propose an algorithm (referred to as the GK algorithm below) that has a worst-case space requirement of $O(\frac{1}{\epsilon}\log(\epsilon N))$, surpassing the MRL algorithm both theoretically and empirically. The general idea of GK algorithm maintains a sorted subset of data input by the combine and prune operations which keep the precision in control. The space complexity is then improved to 
$O(\frac{1}{\epsilon}\log^{\frac{3}{2}}\frac{1}{\epsilon})$ by \citeauthor{agarwalMergeableSummaries2013}\cite{agarwalMergeableSummaries2013}, who provide a mergeable sketch based on the GK algorithm.

A deterministic, fixed-universe algorithm \textit{Q-Digest} was designed by \citeauthor{shrivastavaMediansNewAggregation2004b}\cite{shrivastavaMediansNewAggregation2004b}. It deals with the quantile problem as a histogram problem using $O (\frac{1}{\epsilon}\log U)$ space, where $U$ is the size of the fixed universe from which the input is drawn.


\subsection{Randomised algorithms}
\label{randomised}

Estimating quantile from a random sample of data stream input is the basic idea for most randomized algorithms, and it reduces space usage to a considerable extend.
A simple randomized algorithm designed by \citeauthor{floydExpectedTimeBounds1975}\cite{floydExpectedTimeBounds1975} uses $O(\frac{1}{\epsilon^2} \log \frac{1}{\epsilon \delta})$ space, for an $\epsilon$-approximate $\phi$-quantile of N elements with probability at least $1 - \delta$, where $0 < \delta < 1$.
Later, \citeauthor{mankuRandomSamplingTechniques1999}\cite{mankuRandomSamplingTechniques1999} proposed a randomized quantile estimation algorithm based on their previous framework in \cite{mankuApproximateMediansOther1998}. The new method requires space of size 
$O(
    \frac{1}{\epsilon} \log^2 
    (\frac{1}{\epsilon} 
        \log \frac{1}{\epsilon \delta}
    )
)
$
, being the first practical algorithm that does not require advance knowledge of data stream size $N$.

The \textit{turnstile model} was first taken into consideration by \citeauthor{gilbertChapter40How2002}\cite{gilbertChapter40How2002}, who designed the \textit{random subset sum} sketch with space complexity 
$O(\frac{1}{\epsilon^2} \log^2 U \log (\frac{\log U}{\epsilon}))$, where $U$ is the size of the universe. 
The key idea is based on their findings that insertions or deletions of quantiles can be reduced to finding range sums.
The space requirement is greatly improved by another algorithm for turnstile model -- the \textit{Count-Min sketch} designed by \citeauthor{cormodeImprovedDataStream2005}\cite{cormodeImprovedDataStream2005}. Only 
$O(\frac{1}{\epsilon} \log^2 N \log (\frac{\log N}{\phi \delta}))$ space is needed for an $\epsilon$-approximate $\phi$-quantile. Although the space requirement is worse than the deterministic algorithms \textit{GK} and \textit{Q-Digest}, it has a noticeable advantage on dealing with both insertions and deletions to streams.

Another significant improvement on randomised algorithms were made in 2017, when \citeauthor{felberRandomizedOnlineQuantile2017}\cite{felberRandomizedOnlineQuantile2017} developed an algorithm that uses $O(\frac{1}{\epsilon}\log \frac{1}{\epsilon})$ of memory space. The algorithm not only beats the previous best upper bound of $O(\frac{1}{\epsilon}\log^{\frac{3}{2}}\frac{1}{\epsilon})$ by \citeauthor{agarwalMergeableSummaries2013}\cite{agarwalMergeableSummaries2013}, but it also beats the lower bound for any deterministic comparison-based algorithm, according to the work of \citeauthor{hungLogSpaceLower2010}\cite{hungLogSpaceLower2010}. The estimation ensures the return of the $\epsilon$-approximate quantile summary, with probability at least $1-e^{-poly(1/\epsilon)}$.


\subsection{Algorithms for other requirements}
\label{other}
Due to the large range of data streams and quantile application requirements, it is not surprising to have a considerable variety in quantile estimation algorithms attacking on specific requirements. We will introduce some interesting applications in the following, such as the methods for biased accuracy requirements and the parallel algorithm for distributed data streams.
\\\\
Although most quantile estimation work target at the \textit{uniform accuracy} problem on quantile estimation, many queries over data streams require differently. For example, the $0.99$-quantile is usually sparsely surrounded while the $0.5$-quantile is likely to locate in the center of a distribution where the data is much more dense. Thus comes the \textit{biased quantiles} problem which demands higher accuracy for more extreme quantile values.
The first deterministic algorithms for the biased quantiles problem is proposed by \citeauthor{cormodeSpaceTimeefficientDeterministic2006} \cite{cormodeSpaceTimeefficientDeterministic2006} that takes only space $O(\frac{1}{\epsilon} \log {U} \cdot \log {\epsilon N})$, where $N$ is the size of the data stream and $U$ is the size of universe from which the samples are drawn. Besides, tha algorithms also apply for Other related problems like biased rank queries and targeted quantiles. 

There are also applications requiring algorithms work for distributed system where a big data stream is processed by different processors.
The \citeauthor{ben-haimStreamingParallelDecision2010}\cite{ben-haimStreamingParallelDecision2010} introduces an on-line histogram building algorithm \textit{Streaming Parallel Decision Tree (SPDT)} in which the histogram boundaries can be seen as estimated quantile values.
In this method, multiple histograms are built from streaming data in parallel, which are then merged into a summary histogram of the entire dataset. The summary histogram is a set of sorted real numbers that represents the interval boundaries such that all the intervals have approximately the same size. Specifically, for a summary histogram with $N$ intervals, the set of real numbers is approximately the set of $\tau$-quantiles ($\tau = \frac{1}{N}, \frac{2}{N}, ..., \frac{N-1}{N}$) for the input data stream.
Some quantiles estimation methods mentioned before are also available for distributed settings. To mention a few,
{Q-Digest}\cite{shrivastavaMediansNewAggregation2004b} is a tree based algorithm which can be extended for distributed quantile processors by merging the quantile summaries across different nodes. The Count-Min sketch\cite{cormodeImprovedDataStream2005} is also suited for distributed applications too. Both adjustments do not get involved in change of space complexity.
The {GK} algorithm also has a structure that maintains distributed nodes, and \citeauthor{greenwaldPowerconservingComputationOrderstatistics2004}\cite{greenwaldPowerconservingComputationOrderstatistics2004} found a method to fit it in distributed settings with a larger space complexity $O(\frac{1}{\epsilon}log^3 N)$.
It was \citeauthor{agarwalMergeableSummaries2013}\cite{agarwalMergeableSummaries2013} who made the significant improvement to lower space complexity 
$O(\frac{1}{\epsilon}\log^{\frac{3}{2}}\frac{1}{\epsilon})$
and distributed setting capability.

In the situation when old data get expired, the applications seek for quantile of only the most recent observations from a data stream.
The \textit{sliding window} represents the most recent $W$ in a data stream. The types of sliding window can be classified as \textit{fixed-size} for $W$ is fixed and \textit{variable-size} when $W$ changes.
In \cite{linContinuouslyMaintainingQuantile2004}, \citeauthor{linContinuouslyMaintainingQuantile2004} posed their algorithm for fixed-size sliding window problem with an 
$O(\frac{1}{\epsilon} \log (\epsilon^2 W) + \frac{1}{\epsilon^2})$ space requirement.
The result was then beaten by the work of \citeauthor{arasuApproximateCountsQuantiles2004}\cite{arasuApproximateCountsQuantiles2004} with space complexity 
$O(\frac{1}{\epsilon} \log\frac{1}{\epsilon} \log W)$,
which are designed for both fixed and variable sized windows.
Despite of a worse memory space usage, the work of \citeauthor{linContinuouslyMaintainingQuantile2004}\cite{linContinuouslyMaintainingQuantile2004} has the advantage of estimating quantiles for the most $w$ observations ($w < W$) with $O(\frac{1}{\epsilon^2} \log (\epsilon W))$ space requirement.


    % This summary histogram is notable for its evenly distributed intervals sizes, as each interval has the same number of data points. To interpret the histogram into quantiles, 

% The stochastic gradient descent algorithm, however, is proposed under the assumption that the si` zze of data stream samples is unknown. \textbf{i dont know what to write for the accuracy/convergence part QAQ}
%----------------------------------------------------------------------------------------
%   Single quantile estimation
%----------------------------------------------------------------------------------------

\pagebreak
\fi

\section{Non-growing memory for single quantile estimation}
\label{singlequantile}
The brief history above has shown a clear path for quantile estimation algorithm development: keeping a relatively accurate result while using as little memory as possible. This triggers another interesting question: what if only a constant unit of memory is allowed and how would it affect the accuracy on quantile estimation? In the following part, we introduce two similar algorithms that estimate a single quantile with a $O(1)$ space requirement.
% \textbf{Online algorithms: Data stream algorithms have an online component where input is revealed in steps, but they have resource constraints that are not typically incorporated in competitive analysis of online algorithms. \cite{muthukrishnanDataStreamsAlgorithms2005}}

\citeauthor{maFrugalStreamingEstimating2014}\cite{maFrugalStreamingEstimating2014} introduce the randomized algorithms frugal streaming (referred to as Frugal below) that require only one or two units of memory.
They have an important difference from the randomized algorithms introduced before. 
As mentioned before, the key idea of randomized algorithms is the maintenance of a random set of samplings from the data stream. The Frugal algorithm, however, is not able to keep such a samplings with its extremely limited space, its randomness is caused by another design of the algorithm.

In \textit{Frugal-1U}, the algorithm requires only one unit of memory to record the current quantile estimate. On arrival of each sample, the quantile estimate either stays unchanged or changed by a constant value. 
% An improved version is Frugal-2U, which needs another memory unit for a better convergence rate. 

To attack the same problem, \citeauthor{yazidiQuantileEstimationDynamic2016}\cite{yazidiQuantileEstimationDynamic2016} resort to the idea of online learning and achieves a similar method. 
One distinction is step size. In this algorithm, each step the new estimate is updated by a variable related with the current estimate rather than a constant.
Their work is inspired by \citeauthor{tierneySpaceEfficientRecursiveProcedure1983}\cite{tierneySpaceEfficientRecursiveProcedure1983}, who introduce stochastic learning method to quantile estimation problems. 
% In The noticeable similarity and difference between \citeauthor{yazidiQuantileEstimationDynamic2016}'s work and the Frugal-1U is analysed in \textbf{section ??}.  

In \textbf{section ?}, we provide a theoretical analysis and experiment comparisons on the noticeable similarity between the Frugal-1U and \citeauthor{yazidiQuantileEstimationDynamic2016}'s work . 

%----------------------------------------------------------------------------------------
%   Multi- quantile estimation
%----------------------------------------------------------------------------------------
\iffalse
\section{Multi-quantile estimation}
\label{multiquantile}
When more quantile numbers are required, for example, two quantiles each identifying the upper and lower outlier of a distribution, the single quantile estimation methods become inefficient and less applicable.
For only a few quantiles, this problem can be solved by running parallel single quantile estimation processes for each quantile. But the issue remains when the number of processes excesses the computer's parallel capacity.
The algorithm which estimates several quantile values in one process, as a general solution to this problem, is then brought forward.
Multiple quantile estimation is the simultaneous estimation on different quantile values from streaming data. 
In general, it is the estimator that estimates $k$ quantiles (the $\tau_1$-, $\tau_2$-, $...$, $\tau_k$-quantiles) at the same time.
Given the relationship between 
It has been an issue targeted by different algorithms.
        % excluded: \cite{pebayFormulasRobustOnepass2008}: online learning, not quantile\\\\
    \subsection{Other Method and Advantages of simultaneous estimation\cite{mcdermottDataSkeletonsSimultaneous2007}}

    Another single-pass low-memory methods for simultaneous multi-quantile estimation is the Data Skeleton(DS)\cite{mcdermottDataSkeletonsSimultaneous2007} algorithm, which is derived from the method proposed by \citeauthor{liechtySinglepassLowstorageArbitrary} \cite{liechtySinglepassLowstorageArbitrary}. For an estimation of $k$ quantiles, the algorithm requires the first $km$ data points($m$ is a constant) being sorted, and updates this tracking array on each new observation. Instead of the $k$ estimates of quantiles, it returns a total of $km$ estimates due to the redundancy of computation. This feature is considered an advantage for certain applications like density estimation, when extra quantile estimates is useful in accuracy improvement.


    Over the comparison with \citeauthor{liechtySinglepassLowstorageArbitrary}'s algorithm, \citeauthor{mcdermottDataSkeletonsSimultaneous2007}\cite{mcdermottDataSkeletonsSimultaneous2007} find simultaneous estimation on multiple quantiles has two main advantages over single quantile estimation methods. First is the save in computation time as it does not need to estimate quantiles separately. The second advantage, according to the experiments, is the accuracy improvement in simultaneous quantile estimation.  


    
\textbf{
    Parallel quantile estimation using Stochastic approximation \cite{hammerSmoothEstimatesMultiple2019}
}

\textbf{P2 algorithm and latter algorithms that based on it}


% \section{Anomaly Detection and Outlier}
\fi