% \documentclass[11pt]{article}

\chapter{Background}
\label{ch: background}
% \begin{document}
% \maketitle
In this chapter, the essential background knowledge of the usage of stochastic gradient descent and definition of quantiles are introduced respectively in section\ref{sec: sgd} and\ref{sec: quant}. Stochastic gradient descent (SGD) is a convex optimization approach commonly applied in machine learning, and quantiles are the dividing points of a distribution or a dataset.

\section{Gradient descent and Stochastic Gradient Descent}
\label{sec: sgd}
    Stochastic gradient descent (often abbreviated SGD) is an optimization algorithm developed from gradient descent. 
    In this section, gradient descent is introduced as the first part of the explanation of SGD.

    \subsection{Gradient Descent}
        For convex optimization problems, gradient descent is a first-order optimization algorithm 
        to find the local minimum of a function.
        \\\\
        To solve the minimization problem 
        \begin{equation}
            % E
            \min_{\x} L(\x) 
        \end{equation} 
        
        where $L : \R^d \to \R$ is convex, differentiable and its gradient is Lipschitz continuous with constant
        $L > 0$.
        \\\\
        Geometrically, the gradient $\nabla L(\x_0)$ points to the direction of the steepest ascent on $L(\cdot)$ 
        from the point $\x_0$. 
        By taking a small step in the direction of the negative gradient, the function value is decreased in the 
        direction of the steepest descent. That is,
        \begin{equation}
            \x_1  = \x_0 - \alpha \nabla L(\x_0)
        \end{equation}
    
        for a small enough step size $\alpha \in \R_{+}$, then $L(\x_1) \leq L(\x_0)$. 
        That means, compared with $L(\x_0)$, $L(\x_1)$ is closer to the local minimum.
        \\\\
        With this observation comes the idea of gradient descent: an iterative "tour" on $L(\cdot)$ from a point towards the 
        local minimum by following small steps of negative gradient. 
        Let $\x_0$ be the guess of a starting point, then if
        \begin{equation}
            \x_{k+1} = \x_{k} - \alpha_k \nabla L(\x_k), k \geq 0
        \end{equation}
        
        
        Then we have $ L(\x_0) \geq L(\x_1) \geq L(\x_2) \geq \cdots$ with suitable $\alpha_k$. The convergence rate of the 
        sequence $(\x_n)$ with certain step size settings is linear in terms of the number of iterations.
        Gradient descent, however, requires access to all the data points for the computation of a the moving direction, which is infeasible for data stream situations.


    \subsection{Stochastic Gradient Descent (SGD)}
        Stochastic gradient descent is the choice for data streams, for the computation of the moving direction relies only on the most recent coming data point.
        It can be considered as a stochastic approximation of gradient descent optimization, 
        when the objective function $L(\cdot)$ can be written as a sum of differentiable functions.
        Consider the objective function is in the form:
        \begin{equation}
            L(\x) =\frac{1}{K} \sum_{k=1}^{K} L_k (\x)
        \end{equation}

        where the \textit{summand function} $L_k$ is usually the loss function of the $k$th observation among
        $K$ data points.
        \\\\
        Then by following the idea of gradient descent, the $\x$ is updated according to
        
        \begin{equation}
           \x_{k+1} = \x_{k} -\alpha_k \nabla L(\x_k) = \x_{k} -\alpha_k \frac{1}{K}\sum_{k=1}^{K} \nabla L_k(\x_k) 
        \end{equation}
        
        
        where each $\alpha_k$ is a suitable step size. The calculation of $\sum_{k=1}^{K} \nabla L_k(\x_k)$ can be
        expensive, especially when the amount of summand functions is huge, or when the individual gradients are hard to
        compute. 
        \\\\
        To reduce the consumption of calculation, an estimation of the true gradient of $L(\x)$ is taken: 
        the true gradient $\frac{1}{K} \sum_{k=1}^{K} \nabla L_k(\x_k)$ is replaced by the gradient of a single observation $\nabla L_k(\x_k)$. 
        So the update of the parameter $\x$ becomes
        
        \begin{equation}
            \x_{k+1} = \x_{k} - \alpha_k \nabla L_k(\x_k)
        \end{equation}
        
        where $\alpha_k$ is a suitable step size. 
        \\\\
        While the cost of per update computation is decreased by $N$ times, the tradeoff lies in the convergence rate such that SGD does not enjoy the linear convergence rate like gradient descent.

\subsection{Discussion}
\marginpar{Not sure how I should mention this non-differentiable issue}
Since the each SGD update requires only the coming data instead of the entire data, it becomes a great alternative for quantile estimation if the corresponding loss function meets the requirement.
However, SGD cannot handle non-differentiable functions. This is an issue with the quantile estimation loss function in the following chapter.


\section{Quantile}
\label{sec: quant}

In statistics, quantiles are the points that divide a probability distribution into even intervals.
The $q$-quantiles ($q = 2,3,4,...$) divide the distribution into $q$ intervals each with the same amount of data points.
And there are $q$ quantile points of the $q-$quantiles.
For example, the $2$-quantile has only one quantile point, which is the middle point of the distribution
and it divides the distribution into two even parts. This $2$-quantile point is called the median.
        
        
    \subsubsection{Definition with notation $q$} \label{q-quantile-def}
    Generally, the $q$-quantiles have $q-1$ quantile points, and the $k$th $q$-quantile for a 
    distribution $X$ is the data value such that
    
    \begin{equation}
        Pr(X \leq x) \geq \frac{k}{q}
    \end{equation}
    
    and
    
    \begin{equation}
        Pr(X \geq x) \geq 1 - \frac{k}{q}
    \end{equation}
    where $x \in X$.

    \subsubsection{Definition with notation $\tau$} \label{tau-quantile-def}
    Since $k < q$, we have $0 < \frac{k}{q} < 1$ for all ($k$,$q$) pairs. And the infinite number of ($k$,$q$) pairs can reach any point in interval $(0,1)$. For a general representation, we use the notation $\tau$-quantile ($0 < \tau < 1$) for a quantile value in the distribution $X$ that satisfies

    \begin{equation}
        Pr(X \leq x) \geq \tau
    \end{equation}
    
    and
    
    \begin{equation}
        Pr(X \geq x) \geq 1 - \tau
    \end{equation}
    where $x \in X$.